% !TeX root = ../thuthesis-example.tex

% 中英文摘要和关键字

\begin{abstract}
   (TODO: autotranslated) 最近,在构建物理上可实现的探测器规避攻击方面取得了巨大成功。
   尽管该领域取得了快速进展,但物理领域的对抗性攻击仍然常常不切实际:人眼很容易发现对抗性示例,因为它们不寻常的鲜艳颜色或过于显眼的怪异图案。
  
   在这篇硕士论文中,我们提出了一种方法来创建真实的对抗纹理以逃避检测器,但在性能方面仍可与最先进的检测器竞争。
   受扩散概率模型强大生成能力的启发,我们采用预训练模型,通过指导生成物理上可实现的服装对抗纹理。
  
   论文通过将结果与基线方法进行比较,证明了所提出的方法生成的模式的有效性、自然性和可控性。
   虽然模式的对抗效应并不高于最先进的方法,但我们的策略可以控制模式并能够以自然性来交换对抗效应。
   此外,由于该方法适用于预训练模型,因此它比基线要快得多。

  % 关键词用“英文逗号”分隔,输出时会自动处理为正确的分隔符
  \thusetup{
    keywords = {对抗性攻击、物体识别、扩散模型、物理攻击、指导},
  }
\end{abstract}

\begin{abstract*}
  Recently, huge success has been achieved in constructing physically realizable evasion attacks on detectors.
  Despite the rapid progress in the area, adversarial attacks in the physical domain still often suffer from being unrealistic: the human eye can easily spot adversarial examples due to their unusual vivid colors or overly conspicuous bizarre patterns.
  
  In this master thesis, we propose a method to create realistic adversarial textures for evading detectors that would nevertheless compete with the state-of-the-art ones in terms of performance.
  Motivated by the strong generation abilities of diffusion probabilistic models, we employ pretrained models for generation of physically realizable adversarial texture for clothing using guidance.
  
  The thesis demonstrates the effectiveness, the naturalness, and the controlability of the patterns generated with the proposed method by comparing the results against baseline methods.
  While the adversarial effect of the patterns is not higher than that of the state-of-the-art methods, our strategy gives control over the pattern and enable trading adversarial effects off for naturalness.
  Moreover, since the method works on pretrained models, it is significantly faster than the baselines.

  % Use comma as separator when inputting
  \thusetup{
    keywords* = {adversarial attacks, object detection, diffusion models, physical attacks, guidance},
  }
\end{abstract*}
