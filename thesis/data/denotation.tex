% !TeX root = ../thuthesis-example.tex

\begin{denotation}[3cm]
  \item[AP] Average Precision
  \item[ASR] Attack Success Rate
  \item[bbox] Bounding BOX
  \item[CLIP] Contrastive Language-Image Pre-training
  \item[CNN] Convolutional Neural Network
  \item[COCO] Common Objects in COntext, Microsoft dataset for object detection 
  \item[CV] Computer Vision
  \item[DETR] DEtection TRansformer, a transformer-based object detection model 
  \item[DL] Deep Learning
  \item[DM] Diffusion Model
  \item[DNN] Deep Neural Network
  \item[IoU] Intersection over Union
  \item[LM] Language Model
  \item[ML] Machine Learning
  \item[R-CNN] Regions with CNNs, a region-based object detection model 
  \item[RL] Reinforcement Learning
  \item[TPS] Thin plate splines, augmentation technique for imitating clothes
  \item[VAE] VAriational Encoder
  \item[YOLO] You Only Look Once, a grid-based object detection model 
\end{denotation}



% 也可以使用 nomencl 宏包,需要在导言区
% \usepackage{nomencl}
% \makenomenclature

% 在这里输出符号说明
% \printnomenclature[3cm]

% 在正文中的任意为都可以标题
% \nomenclature{PI}{聚酰亚胺}
% \nomenclature{MPI}{聚酰亚胺模型化合物,N-苯基邻苯酰亚胺}
% \nomenclature{PBI}{聚苯并咪唑}
% \nomenclature{MPBI}{聚苯并咪唑模型化合物,N-苯基苯并咪唑}
% \nomenclature{PY}{聚吡咙}
% \nomenclature{PMDA-BDA}{均苯四酸二酐与联苯四胺合成的聚吡咙薄膜}
% \nomenclature{MPY}{聚吡咙模型化合物}
% \nomenclature{As-PPT}{聚苯基不对称三嗪}
% \nomenclature{MAsPPT}{聚苯基不对称三嗪单模型化合物,3,5,6-三苯基-1,2,4-三嗪}
% \nomenclature{DMAsPPT}{聚苯基不对称三嗪双模型化合物(水解实验模型化合物)}
% \nomenclature{S-PPT}{聚苯基对称三嗪}
% \nomenclature{MSPPT}{聚苯基对称三嗪模型化合物,2,4,6-三苯基-1,3,5-三嗪}
% \nomenclature{PPQ}{聚苯基喹噁啉}
% \nomenclature{MPPQ}{聚苯基喹噁啉模型化合物,3,4-二苯基苯并二嗪}
% \nomenclature{HMPI}{聚酰亚胺模型化合物的质子化产物}
% \nomenclature{HMPY}{聚吡咙模型化合物的质子化产物}
% \nomenclature{HMPBI}{聚苯并咪唑模型化合物的质子化产物}
% \nomenclature{HMAsPPT}{聚苯基不对称三嗪模型化合物的质子化产物}
% \nomenclature{HMSPPT}{聚苯基对称三嗪模型化合物的质子化产物}
% \nomenclature{HMPPQ}{聚苯基喹噁啉模型化合物的质子化产物}
% \nomenclature{PDT}{热分解温度}
% \nomenclature{HPLC}{高效液相色谱(High Performance Liquid Chromatography)}
% \nomenclature{HPCE}{高效毛细管电泳色谱(High Performance Capillary lectrophoresis)}
% \nomenclature{LC-MS}{液相色谱-质谱联用(Liquid chromatography-Mass Spectrum)}
% \nomenclature{TIC}{总离子浓度(Total Ion Content)}
% \nomenclature{\textit{ab initio}}{基于第一原理的量子化学计算方法,常称从头算法}
% \nomenclature{DFT}{密度泛函理论(Density Functional Theory)}
% \nomenclature{$E_a$}{化学反应的活化能(Activation Energy)}
% \nomenclature{ZPE}{零点振动能(Zero Vibration Energy)}
% \nomenclature{PES}{势能面(Potential Energy Surface)}
% \nomenclature{TS}{过渡态(Transition State)}
% \nomenclature{TST}{过渡态理论(Transition State Theory)}
% \nomenclature{$\increment G^\neq$}{活化自由能(Activation Free Energy)}
% \nomenclature{$\kappa$}{传输系数(Transmission Coefficient)}
% \nomenclature{IRC}{内禀反应坐标(Intrinsic Reaction Coordinates)}
% \nomenclature{$\nu_i$}{虚频(Imaginary Frequency)}
% \nomenclature{ONIOM}{分层算法(Our own N-layered Integrated molecular Orbital and molecular Mechanics)}
% \nomenclature{SCF}{自洽场(Self-Consistent Field)}
% \nomenclature{SCRF}{自洽反应场(Self-Consistent Reaction Field)}
